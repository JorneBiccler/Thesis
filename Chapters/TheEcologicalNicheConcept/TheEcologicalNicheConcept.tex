\chapter{The ecological niche concept}
\label{ch:TheEcologicalNicheConcept}
\section{Introduction}
\label{sec:chTheEcologicalNicheConcept:Introduction}
In this chapter the concept of the niche of a species is introduced. We can, non-rigorously, define the ecological niche of a species as the set of environmental conditions where its reproduction rate is larger or equal to its mortality rate. Although we will speak of the ecological niche, there are in fact at least three different ``definitions'' that are often used: the Grinnellian niche, the Eltonian niche, and the Hutchinsonian niche. Only a sketch of the niche concept will be given in this section. For a more rigorous desription we refer the interested reader to \cite{soberon_grinnellian_2007} and, \cite{soberon_niches_2009}.

\section{Ecological versus geographical space}
In most databases that contain data about species only the location of a presence or absence record is available. Hence, these databases include information about the occurrences or absences in the so-called geographical space. Usually the range of a species distribution is determined by environmental conditions. We will say that the corresponding variables span the environmental space. It is clear that for each point in the geographical space there is a point in the environmental space. This relation between environmental and geographical space is often called Hutchinson's duality \parencite{colwell_hutchinsons_2009}. A graphical representation of this relation is given in Figure \ref{fig:chTheEcologicalNicheConcept:Niche}. This duality relation is fundamental in SDMs, namely the predictors in the model are usually assumed to be direct or indirect measures of the variables that span the environmental space. Once a model in the environmental space is constructed the duality relation allows us to make maps of the distribution in the geographical space. \\

\begin{figure}[!htb]
\includegraphics[scale=0.5]{VectorGraphics/Niche.png}
\caption{\label{fig:chTheEcologicalNicheConcept:Niche}Visualization of the duality between environmental and geographical space.}
\end{figure}

In practice a species will often not occur in certain parts of its niche. This can happen because of limited dispersal capabilities of the species, biotic interactions, etc.\ Such incomplete occupation of the niche leads to the concepts of a fundamental niche and the realized niche. The fundamental niche does not take into account whether or not the species is present, it only represents the suitable conditions. The realized niche is the subset of the fundamental niche where the species is present. These two concepts are depicted in Figure \ref{fig:chTheEcologicalNicheConcept:RealizedNiche}. 
\begin{figure}[!htb]
\includegraphics[scale=0.5]{VectorGraphics/RealizedNiche.png}
\caption{\label{fig:chTheEcologicalNicheConcept:RealizedNiche}Visualization of the difference between the fundamental and realized niche.}
\end{figure}

\section{Implicit assumptions when building and using species distribution modles}
\label{sec:chTheEcologicalNicheConcept:NicheEquilibrium}
Before applying SDMs in practice we have to realize that there are some important underlying assumptions. We will only describe a few of these assumptions. This is done in order to connect the theoretical niche concept to some more practical scenarios and to make the reader aware of the limitations of species distribution modelling. For a more complete overview of the underlying assumptions we refer to \cite{wiens_niches_2009}.\\

By definition every observation in the field belongs to the realized niche. SDMs are therefore models of the realized niche. If we want to use a certain SDM to e.g.\ predict areas prone to invasive species, we implicitly assume that, a part of, the realized niche is a good approximation of, a part of, the fundamental niche. Whether this assumption is realistic or not depends on: the species, whether the whole niche has to be approximated or only a part thereof, etc.\\

When we use data to build a model of the realized niche we have to assume that the observed data-points are representative of the niche. In practical settings this is often not the case. For example, due to climate change tree species might be found in regions where the current environmental conditions are not included in its niche. The niche of a species usually evolves over time. This is another reason of why the observations may not be representative of the niche we are modelling. Hence, we have to assume that historical records are representative of the niche that is being modelled. \\

Another assumption implicitly made in most SDMs is that the effect of biotic interactions is negligible or indirectly captured by other environmental variables. However, in some applications explicitly including biotic interactions has been shown to improve the predictive capabilities \parencite{heikkinen_biotic_2007}. \\






