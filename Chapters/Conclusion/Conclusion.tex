\chapter{Conclusion}
\label{ch:Conclussion}
\section{}

\section{Future research}
Although the results in this thesis give some indication of the performance of variable selection methods the number of studied species is too small to make conclusive recommendations. Investigating the performance of the different models for additional species, preferably also in different continents, should solidify the conclusions.\\

It seems that introducing new statistical techniques to the field of SDM could lead to significant improvements over current practices. \\

Given that it is quite hard to manually check whether observations are outliers or not, see Chapter \ref{ch:DataCommonlyUsedInSpeciesDistributionModels} methods to deal with outliers could be of interest. In the statistical literature methods that are not (heavily) influenced by outliers are called robust. Robust variations of GLMs and IPP models are already available \parencite{cantoni_robust_2001, assuncao_robustness_1999}. It could be interesting to adapt these to presence-only modelling. E.g. if a robust GLM is used in combination with background data perhaps the method can be adapted such that background points are never considered to be outliers. Furthermore, the connections between MaxEnt, IPP models, and GLMs should make it doable to use the robust variations of GLMs and IPP models to construct a robustified MaxEnt method. Finally, also model selection is influenced by outliers recently some methods have become available that deal with this scenario \parencite[e.g.][]{muller_robust_2009} \\

In Section \ref{sec:CaseControlSubsampling} sub-sampling of presence-absence data was introduced. Instead of taking a completely random subsample of the absence points more advanced methods exist \parencite{king_logistic_2001}. Although the gain of these methods might turn out to be minimal it could open the door to using even more advanced methods that are as of yet too computationally intensive. \\

Even though we tried to study a number of different variable selection techniques a lot of different approach are available or being developed. Examples include the focused information criterion \parencite[FIC, ][]{claeskens_focused_2003}, the elastic net penalty \parencite{zou_regularization_2005}, \dots\ The FIC is an information criterion that can be adjusted with the goal of the study in mind. E.g.\ a FIC that focusses on variable selection with as goal the predictions of occurrence probability in a few select locations can be constructed. The elastic net is a new penalization technique that combines the $L_2$ and $L_1$ penalties and has been shown to perform quite well.\\

