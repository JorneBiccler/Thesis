\chapter{Conclusion}
\label{ch:Conclussion}
\section{Discussion and overview of the results}
After introducing the different classification, variable selection, and dimensionality reduction methods and their implementations in Chapters \ref{ch:ClassificationTechniques}, \ref{ch:ReducingTheNumberOfExplanatoryVariables}, and \ref{ch:Implementations} they were applied to real and simulated data in Chapters \ref{ch:Applications} and \ref{chap:SimulationStudy}. The main conclusions from the two previous chapters are summarized below. \\

Chapter \ref{ch:Applications} started with Section \ref{sec:AUC} in which the use of the AUC as a performance measure in SDM was shortly discussed and deemed appropriate for our purposes. In Section \ref{sec:POData} $10$ species and $20$ classification methods were studied. The results can be summarized as:
\begin{itemize}
\item A simple logistic regression model with only linear terms can perform quite well.
\item Flexible methods that allow for non-linear relations tend to outperform logistic regression.
\item Our implementations of penalized logistic regression performs on par with the standard logistic regression method.
\item MaxEnt does not seem to perform as well as its popularity would imply, especially when the default parameters are used other methods tend outperform it.
\end{itemize}

In order to deal with the large amount of data available from the FIA database we introduced sub-sampling as a method to speed up the computations in Section \ref{sec:CaseControlSubsampling}. As far as we are aware this was the first time this type of sub-sampling was used in the case of presence-absence SDM. By using this sub-sampling we were able to facilitate the fitting of $14$ classification methods on the FIA data-sets. The results were presented in Section \ref{sec:PAResults} and it could be concluded that roughly the same trends in classification performance were observed as when presence-only data was used. \\

Based on these results it was decided to only perform a simulation study for the presence-only scenario. First of all, this is the most popular scenario in practice. Secondly, since roughly the same classification performance was observed when presence-absence data is used it seems reasonable that the results of a presence-only simulation study carry over to some extent. \\

The results of the simulation study mainly confirmed the conclusions of Chapter \ref{ch:Applications}. The exception is that the logistic07 selection method tends to be quite variable which is behaviour that was only observed in the case of the presence-absence data and not in the case of the presence-only data. Given that two out of the three studies imply that the select07 logistic regression method does not perform well we advise against using it. Again the select07 logistic regression method seems to be quite popular as a pre-screening technique. \\

Given the results of the studies on the real and simulated data we conclude that MaxEnt might be somewhat overused throughout SDM. Especially since the use of the default parameters is rampant this seems somewhat inappropriate. Perhaps the most important reason of the popularity of the MaxEnt method is that it is provided as a standalone program that is easy to use. Hence, it might be interesting to perhaps develop a graphical user interface (GUI) that allows the user to provide rasters and allow for multiple different models to be fitted in a point and click fashion.

\section{Future research}
Although the results in this thesis give some indication of the performance of variable selection methods the number of studied species is too small to make conclusive recommendations. Investigating the performance of the different models for additional species, preferably also in different continents, should solidify the conclusions. Furthermore, by enlarging the simulation study, especially considering more virtual species might be interesting, more conclusive results can probably be obtained.\\

It seems that introducing new statistical techniques to the field of SDM could lead to significant improvements over current practices. \\

Given that it is quite hard to manually check whether observations are outliers or not, see Chapter \ref{ch:DataCommonlyUsedInSpeciesDistributionModels} methods to deal with outliers could be of interest. In the statistical literature methods that are not (heavily) influenced by outliers are called robust. Robust variations of GLMs and IPP models are already available \parencite{cantoni_robust_2001, assuncao_robustness_1999}. It could be interesting to adapt these to presence-only modelling. E.g. if a robust GLM is used in combination with background data perhaps the method can be adapted such that background points are never considered to be outliers. Furthermore, the connections between MaxEnt, IPP models, and GLMs should make it doable to use the robust variations of GLMs and IPP models to construct a robustified MaxEnt method. Finally, also model selection is influenced by outliers recently some methods have become available that deal with this scenario \parencite[e.g.][]{muller_robust_2009}. \\

In Section \ref{sec:CaseControlSubsampling} sub-sampling of presence-absence data was introduced. Instead of taking a completely random subsample of the absence points more advanced methods exist \parencite{king_logistic_2001}. Although the gain of these methods might turn out to be minimal it could open the door to using even more advanced methods that are as of yet too computationally intensive. \\

Even though we tried to study a number of different variable selection techniques a lot of different approach are available or being developed. Examples include the focused information criterion \parencite[FIC, ][]{claeskens_focused_2003}, the elastic net penalty \parencite{zou_regularization_2005}, \dots\ The FIC is an information criterion that can be adjusted with the goal of the study in mind. E.g.\ a FIC that focusses on variable selection with as goal the predictions of occurrence probability in a few select locations can be constructed. The elastic net is a new penalization technique that combines the $L_2$ and $L_1$ penalties and has been shown to perform quite well.\\

