\chapter{Introduction}
\label{ch:Introduction}
Species distribution modelling (SDM)\footnote{The abbreviation SDM will be used for both the verb, species distribution modelling, and the noun, species distribution model.} concerns the practice of modelling the distribution of a species by use of explanatory variables. Applications include predicting the effect of climate change \parencite{pearson_predicting_2003, pearson_modelling_2004}, predicting the impact of invasive species \parencite{strubbe_predicting_2008}, predicting the occurrence of wildfires \parencite{parisien_environmental_2009}, \ldots \\

Some fundamental ecological concepts are introduced in Chapter \ref{ch:TheEcologicalNicheConcept}. Firstly, the concept of an ecological niche is introduced and the connection with SDMs is made. Secondly, to enhance the understanding of the niche concept some of the assumptions underlying the practice of SDM are discussed. \\

The data-sets and variables that are used in this thesis are described in Chapter \ref{ch:DataCommonlyUsedInSpeciesDistributionModels}. These variables describe either the climate, human influence, elevation, \dots\ at a certain location. Climate data is nearly always used to model the occurrence probability of species. When the goal of a study is to obtain coarse grain predictions over a large spatial extent the use of climate data is certainly justified by ecological theory \parencite{pearson_predicting_2003}. However, if there is interest in predictions over a relatively small extent, e.g.\ when selecting the location of a new national park, fine grain remote sensing data might be useful to distinguish between suitable and unsuitable habitat. \\

Chapter \ref{ch:ClassificationTechniques} introduces a number of modelling techniques that are often used to model the distribution of species. In Section \ref{sec:PresenceAbsenceData} we focus on classical binary classification methods. These methods are directly applicable if there is access to presence-absence data. However, often the data-sets only includes occurrence locations, for example data-sets from natural history musea or citizen science projects are usually of this type. To use presence-only data the classical classification algorithms from Section \ref{sec:PresenceAbsenceData} can be adapted, this is done in Section \ref{sec:ClassificationWithPseudoAbsences}. Another approach is to use one of the algorithms specifically constructed for presence-only data, one of these is introduced in Section \ref{sec:MaximumEntropyModeling}. \\ 
 
In practice a large part of constructing SDMs consists of variable selection. The goal of this thesis is to investigate the performance of multiple model selection methods. An overview of often used methods to deal with large magnitudes of predictors is given in Chapter \ref{ch:ReducingTheNumberOfExplanatoryVariables}. More particularly, we will introduce: 
\begin{itemize}
\item Regularization. 
\item Step-wise selection. 
\item Dimensionality reduction of the explanatory variables. 
\end{itemize}


Although we will introduce the most important concepts and some applications of species distribution modelling, it is not the goal of this thesis to describe every aspect in detail. Instead we refer to \cite{franklin_mapping_2009} who gave an overview of the field of species distribution modelling. Other introductory material include
\cite{guisan_predictive_2000}, \cite{guisan_predicting_2005}, and \cite{elith_species_2009}. An introduction to most of the statistical methodology can be found in \cite{hastie_elements_2009}.





















