\chapter{Introduction}
\label{ch:Introduction}
Species distribution modelling (SDM)\footnote{The abbreviation SDM will be used for both the verb, species distribution modelling, and the noun, species distribution model.} concerns the practice of modelling the distribution of a species by use of explanatory variables. Applications include predicting the effect of climate change \parencite[e.g.][]{pearson_predicting_2003, pearson_modelling_2004}, the impact of invasive species, the occurrence of wildfires \parencite{parisien_environmental_2009}, \ldots \\

Some fundamental ecological concepts are introduced in Chapter \ref{ch:TheEcologicalNicheConcept}. Firstly, the concept of an ecological niche is introduced and the connection with SDMs is made. Secondly, to enhance the understanding of the niche concept some of the assumptions underlying SDM are discussed. \\

The data-sets and variables that will be used throughout this thesis are described in Chapter \ref{ch:DataCommonlyUsedInSpeciesDistributionModels}. The variables that are used in this thesis describe either the climate at a certain location or are derived from remotely sensed products. Climate data is nearly always used to model the occurrence probability of species. When the goal of a study is to obtain coarse grain predictions over a large spatial extent the use of climate data is certainly justified by ecological theory \parencite{pearson_predicting_2003}. However, if there is interest in predictions over a relatively small extent, e.g.\ when selecting the location of a new national park, fine grain remote sensing data might be useful to distinguish between suitable and unsuitable habitat. \\

Chapter \ref{ch:ClassificationTechniques} introduces a number of modelling techniques that are often used to model the distribution of species. In Section \ref{sec:PresenceAbsenceData} we focus on using data-sets that consist of locations where the species was either present or absent. In this case standard classification methods can be utilized. However, often the data-set only includes occurrence locations, for example data-sets from natural history musea or citizen science projects are usually of this type. To use occurrence only data the classical classification algorithms form Section \ref{sec:PresenceAbsenceData} can be adapted, this is done in Section \ref{sec:ClassificationWithPseudoAbsences}. Another approach is to use one of the algorithms specifically constructed for presence only data, one of these is introduced in Section \ref{sec:MaximumEntropyModeling}. \\ 
 
In practice a large part of building SDMs consists of variable selection. The goal of this thesis is to investigate the performance of multiple model selection methods in settings that are representative of what could be expected in practice. In Chapter \ref{ch:ReducingTheNumberOfExplanatoryVariables} we give an overview of often used methods to deal with large magnitudes of predictors. More particularly, we will introduce: 
\begin{itemize}
\item Regularization. 
\item Step-wise selection. 
\item Dimensionality reduction of the explanatory variables. 
\item So-called ``folklore'' methods.
\end{itemize}


Finally, although we will introduce the most important concepts and some applications of species distribution modeling it is not the goal of this thesis to describe every aspect in detail. Instead we refer to \cite{miller_modeling_2002} who gave an overview of the field of species distribution modeling. Other introductory material include
\cite{guisan_predictive_2000}, \cite{guisan_predicting_2005}, and \cite{elith_species_2009}. An introduction to most of the statistical methodology can be found in \cite{hastie_elements_2009}.





















